\section{Fehlerbehebung}

Bei gravierenden Fehlern wird automatisch eine Fehlerseite angezeigt und die Administration informiert. 

Kleinere Probleme lassen sich meist selbst beheben. Wenn das Aufklappen von Menüs nicht funktioniert,
aktivieren Sie bitte Javascript. Für die Anmeldung auf der Administrationsseite oder das
Speichern von geplanten/erledigten Prüfungsleistungen oder ausgewählten Veranstaltungen
aktivieren Sie bitte Cookies.

Die Seite sollte in allen halbwegs modernen Browsern funktionieren, inklusive den Konsolenbrowsern \texttt{w3m}
und \texttt{lynx}.

\section{Fehler oder Wünsche melden}

Über den Menüpunkt \frq Kontakt\flq\ ganz unten auf der Seite ist es möglich, mit der Studienberatung oder
mir in Kontakt zu treten. Dazu müssen Sie Ihre Email-Adresse, die Art des Anliegens und das Anliegen selbst
formulieren (die Art des Anliegens bestimmt, wohin die Email geht: bei Technischen Fragen, Wünschen,
Verbesserungsvorschlägen oder Fehlerberichten wählen Sie bitte \frq technischer Natur\flq, sonst
\frq inhaltlicher\flq). Dazu muss gegenenfalls noch eine Sicherheitsabfrage beantwortet werden, die aus
sehr einfachen mathematischen oder geographischen Fragen besteht.

\textit{Jede Email wird gelesen! Es wird um viel Feedback gebeten, da ich die Software so gut wie möglich an
die realen Bedürfnisse von Studenten ausrichten möchte!}

Ich möchte aber darum bitten zu bedenken, dass ich die Software nur als Hobby neben dem Studium betreibe
und daher manchmal etwas länger brauchen kann, um Fehler zu finden oder Wünsche einzubauen.



\section{Urheberrechte \& Danksagungen}

Die Urheberrechte sowohl der Software als auch der Dokumentation liegen bei Norman Koch.

Ich möchte mich außerdem bedanken für die freundliche Hilfe des philosophischen Institutes der Technischen
Universität Dresden, Herrn Prof. Dr. Schönrich,
Herrn Prof. Dr. Rentsch und Herrn Prof. Dr. Hiltscher. Vorallem was das Technische angeht, möchte ich mich für die 
Hilfe von  Herrn Dr. Holm Bräuer und Herrn Norbert Engemaier bedanken.
Auch herzlichen Dank an Frau Gilda Märcz, die mir sowohl im Studium als auch in der Verwaltung
immer sehr freundlich und kompetent geholfen hat.

Ohne die Hilfe, die Bugreports und die freundliche
Art aller Mitarbeiter des Institutes wäre diese Software nie entstanden.

\theendnotes

\section{Quellen}

\DeclareFieldFormat{labelnumberwidth}{---}

\printbibliography[heading=none]

\end{document}
