\documentclass{scrartcl}
\newcommand{\frontorbackendtype}{Backend}
\hyphenation{
	Ab-bil-dung
	Ab-kür-zung
	Ab-schnitt
	Ac-count
	Ac-counts
	Ad-mi-nis-t-ra-ti-on
	Ad-mi-nis-t-ra-to-ren
	Ad-mi-nis-t-ra-tor
	Ad-res-se
	Ad-res-sen
	Al-len
	Al-ler
	Al-ter
	An-fra-ge
	An-fra-gen
	An-lie-gen
	An-lie-gens
	An-mel-dung
	An-sicht
	An-wen-dung
	An-zahl
	An-zei-gen
	Ant-wor-ten
	Ant-wort
	Apa-che
	Ar-beits-spei-cher
	Ar-ten
	Auf-bau
	Auf-ru-fen
	Auf-tau-chen
	Aus-nah-me
	Aus-wahl-box
	Be-dürf-nis-se
	Be-fin-den
	Be-gin-nen
	Be-grün-dung
	Be-nut-zer
	Be-nut-zers
	Be-nut-zung
	Be-rei-che
	Be-rich-te
	Be-schrei-bung
	Be-zeich-nung
	Bei-spiel
	Book-mark
	But-tons
	Da-ten-bank
	De-tails
	Do-ku-ment
	Do-zent
	Ein-füh-rung
	Ein-lei-tung
	Ein-stel-lung
	Ein-trag
	Er-stel-lung
	Ers-tem
	Feh-ler-be-he-bung
	Fein-des-hand
	Fest-le-gung
	For-mat
	Funk-ti-o-nen
	Ge-bäu-den
	Ge-bäu-des
	Ge-heim-hal-tung
	Goo-gle
	Ha-kens
	Hei-rat
	Ih-rem
	In-for-ma-ti-o-nen
	In-s-ti-tu-tes
	In-s-ti-tut
	Klar-text
	Klo-nen
	Kon-takt
	Kon-ven-ti-o-nen
	Letz-tem
	Mo-dul
	Nie-mand
	Nor-bert
	Nor-ma-li-sie-rung
	Ober-flä-che
	Obers-ten
	Pass-wort
	Pla-nung
	Pro-b-le-me
	Prü-fung
	Samm-lung
	Schrift-art
	Se-kun-den-takt
	Se-mes-ters
	Sei-ten-rand
	Stu-di-en-gan-ges
	Stu-di-en-gang
	Stu-di-ums
	Sys-tem
	Ver-an-stal-tung
	Ver-bin-dung
	Ver-fü-gung
	Ver-knüp-fung
	Ver-wal-tung
	Vor-gang
	Vor-le-sung
	Zen-t-ra-len
	Zu-griff
	Zu-ord-nung
	Zu-satz-in-for-ma-ti-o-nen
	ab-ge-legt
	ab-s-t-rak-te
	ab-s-t-rak-ter
	ab-sol-viert
	ad-mi-ni-s-tra-tor
	ad-mi-ni-stra-tion
	ad-ress-en
	adres-se
	ak-tu-el-lem
	ak-tu-ell
	ak-zep-tiert
	al-le-samt
	an-ge-legt
	an-ge-mel-det
	an-ge-zeigt
	ant-wort
	aus-ge-führt
	aus-ge-füllt
	aus-ge-wählt
	aus-wählt
	ba-ckend
	ba-siert
	bar-ri-e-re-frei-es
	bar-ri-e-re-freie
	be-ant-wor-tet
	be-ar-bei-tet
	be-fin-det
	be-han-delt
	be-in-hal-tet
	be-kommt
	be-legt
	be-lie-big
	be-nutzt
	be-ste-hen
	be-ste-hend
	be-steht
	be-vor-zugt
	bie-tet
	ca-chen
	check-box
	da-bei
	da-für
	da-ge-gen
	da-nach
	da-r-auf
	da-r-um
	def-in-ition
	di-rekt
	die-sem
	do-ku-ment
	drin-gend
	edi-tie-ren
	ei-nem
	ein-ge-pflegt
	ein-ge-schränkt
	ein-ma-lig
	ent-fernt
	ent-hält
	er-bracht
	er-hält
	er-laubt
	er-le-digt
	er-scheint
	er-schwert
	er-stellt
	ers-tem
	even-tu-ell
	ex-akt
	fer-tig
	fest-ge-legt
	fin-det
	frei-ge-schal-tet
	funk-ti-o-niert
	ganz-tä-gig
	ge-fil-tert
	ge-fragt
	ge-langt
	ge-loggt
	ge-löscht
	ge-macht
	ge-mel-det
	ge-nannt
	ge-ne-riert
	ge-nutzt
	ge-nü-gend
	ge-ord-net
	ge-plant
	ge-sagt
	ge-schieht
	ge-sen-det
	ge-setzt
	ge-spei-chert
	ge-sucht
	ge-wählt
	ge-währt
	ge-än-dert
	ge-öff-net
	goo-gle
	gräu-li-che-rem
	halb-wegs
	her-ge-stellt
	hi-e-r-ar-chi-scher
	hi-e-r-ar-chisch
	hin-zu-ge-fügt
	häu-fig
	ih-rem
	in-sti-tut
	in-sti-tute
	in-te-r-es-siert
	je-dem
	je-mand
	kei-ne
	kei-nen
	klo-nen
	kom-pe-tent
	kon-sis-tent
	kon-takt
	ku-rze
	lei-tet
	letz-tem
	lis-tet
	ma-nu-ell
	min-des-tens
	mit-ge-teilt
	mod-ule
	na-vi-ga-tion
	nie-mand
	not-wen-dig
	nö-tig
	obe-ren
	obers-ten
	on-li-ne
	op-tion-al
	per-sis-tent
	plan-en
	po-ten-zi-ell
	re-a-len
	re-d-un-dan-ten
	re-la-tiv
	re-le-vant
	run-ter
	sei-nen
	sie-he
	sim-plic-ity
	so-fort
	spe-zi-ell
	spei-chert
	stu-dent-en
	stu-dium
	sys-tem
	trotz-dem
	un-ab-hän-gig
	un-nö-tig
	un-ter-punkt
	un-ter-wegs
	up-da-ten
	ver-birgt
	ver-hin-dert
	ver-knüpft
	ver-linkt
	ver-sam-melt
	ver-schwin-det
	ver-wal-tet
	voll-stän-dig
	weit-ge-hend
	wel-chem
	zen-t-ra-len
	zen-t-ral
	zu-fäl-lig
	zu-ge-ord-net
	zu-ge-sagt
	zu-griff
	zu-sam-men-ge-führt
	Än-de-rung
	Über-sicht
	über-all
	über-prüft
        Ab-kür-zung
        Ab-schnitt
        Ac-count
        Ac-counts
        Ad-mi-nis-t-ra-ti-on
        Ad-mi-nis-t-ra-to-ren
        Ad-mi-nis-t-ra-tor
        Al-len
        Al-ler
        Al-ter
        An-fra-ge
        An-fra-gen
        An-lie-gen
        An-lie-gens
        An-mel-dung
        An-sicht
        An-zahl
        An-zei-gen
        Ant-wor-ten
        Apa-che
        Ar-beits-spei-cher
        Ar-ten
        Auf-bau
        Auf-ru-fen
        Auf-tau-chen
        Aus-nah-me
        Be-dürf-nis-se
        Be-fin-den
        Be-gin-nen
        Be-grün-dung
        Be-nut-zer
        Be-nut-zers
        Be-nut-zung
        Be-rei-che
        Be-rich-te
        Bei-spiel
}


\usepackage{mathtools}
\usepackage{fontawesome}
\usepackage{amsmath}
\usepackage{titlesec}
\usepackage{float}
\usepackage{tikz}
\usepackage{graphicx}
\usepackage[
	citestyle=authortitle-ibid,
	isbn=true,
	url=true,
	backref=true,
	backrefstyle=none,
	pagetracker=true,
	maxbibnames=50,
	defernumbers=true,
	maxcitenames=10,
	backend=bibtex,
	urldate=comp,
	dateabbrev=false,
	sorting=nty,
	ibidtracker=true
]{biblatex}
\bibliography{literatur.bib}
\usepackage{MnSymbol,wasysym}

\usepackage[T1]{fontenc}
\usepackage[utf8]{inputenc}
\usepackage[ngerman]{babel}
\usepackage{wallpaper}
\usepackage{hologo}
\usepackage{manfnt}
\usepackage{scrextend}
\usepackage{framed,color,verbatim}
\definecolor{shadecolor}{rgb}{.9, .9, .9}

\usepackage{polyglossia}
\setmainlanguage[babelshorthands=true]{german}
\usepackage{enumitem}
\setitemize{itemsep=0pt}


\usepackage{eulervm}
\usepackage{beton}
\usepackage{noto}
\usepackage{fontspec}
\usepackage{xltxtra}
\usepackage{xunicode}
\usepackage{fixltx2e}
\usepackage{xunicode}
\defaultfontfeatures{Mapping=tex-text}

\usepackage{url}
\makeatletter
\g@addto@macro{\UrlBreaks}{\UrlOrds}
\makeatother

\usepackage{color}   %May be necessary if you want to color links
\usepackage{hyperref}
\hypersetup{
	    colorlinks=false, %set true if you want colored links
	    linktoc=all,     %set to all if you want both sections and subsections linked
	    linkcolor=blue,  %choose some color if you want links to stand out
	    hidelinks
}

\emergencystretch2em
\usepackage{longtable}
\usepackage{etoolbox}

\newcommand{\centeredquote}[2]{
	\hbadness=5000
	\vspace{-1em}
	\begin{flushright}
		\item\frqq\textsl{#1}\flqq\ 
	\end{flushright}
	\nopagebreak
	\hfill ---\,\textsc{#2}\newline
	\vspace{-1em}
}



\newcommand{\dollar}{}
\newcommand{\mydbend}{{\textdbend}}

\usepackage{endnotes}
\let\footnote=\endnote

\newenvironment{technisches}{\par\medskip\snugshade\begingroup\begin{addmargin}[2em]{0em}\mydbend\ }{\end{addmargin}\endgroup\medskip\endsnugshade}

\title{Dokumentation zum \frontorbackendtype\ des VVZs}
\author{Norman Koch}

\setcounter{secnumdepth}{6}

\titleclass{\subsubsubsection}{straight}[\subsection]

\newcounter{subsubsubsection}[subsubsection]
\renewcommand\thesubsubsubsection{\thesubsubsection.\arabic{subsubsubsection}}
\renewcommand\theparagraph{\thesubsubsubsection.\arabic{paragraph}} % optional; useful if paragraphs are to be numbered

\titleformat{\subsubsubsection}
  {\normalfont\normalsize\bfseries}{\thesubsubsubsection}{1em}{}
  \titlespacing*{\subsubsubsection}
  {0pt}{3.25ex plus 1ex minus .2ex}{1.5ex plus .2ex}

  \makeatletter
  \renewcommand\paragraph{\@startsection{paragraph}{5}{\z@}%
    {3.25ex \@plus1ex \@minus.2ex}%
      {-1em}%
        {\normalfont\normalsize\bfseries}}
	\renewcommand\subparagraph{\@startsection{subparagraph}{6}{\parindent}%
	  {3.25ex \@plus1ex \@minus .2ex}%
	    {-1em}%
	      {\normalfont\normalsize\bfseries}}
	      \def\toclevel@subsubsubsection{4}
	      \def\toclevel@paragraph{5}
	      \def\toclevel@paragraph{6}
	      \def\l@subsubsubsection{\@dottedtocline{4}{7em}{4em}}
	      \def\l@paragraph{\@dottedtocline{5}{10em}{5em}}
	      \def\l@subparagraph{\@dottedtocline{6}{14em}{6em}}
	      \makeatother

	      \setcounter{secnumdepth}{5}
	      \setcounter{tocdepth}{5}

%\linespread{1.16}


\newcommand{\ExternalLink}{%
    \tikz[x=1.2ex, y=1.2ex, baseline=-0.05ex]{% 
        \begin{scope}[x=1ex, y=1ex]
            \clip (-0.1,-0.1) 
                --++ (-0, 1.2) 
                --++ (0.6, 0) 
                --++ (0, -0.6) 
                --++ (0.6, 0) 
                --++ (0, -1);
            \path[draw, 
                line width = 0.5, 
                rounded corners=0.5] 
                (0,0) rectangle (1,1);
        \end{scope}
        \path[draw, line width = 0.5] (0.5, 0.5) 
            -- (1, 1);
        \path[draw, line width = 0.5] (0.6, 1) 
            -- (1, 1) -- (1, 0.6);
        }
}

\newcommand*{\fullref}[1]{\hyperref[{#1}]{%\autoref*{#1}
\ExternalLink\textit{\nameref*{#1}} (\ref{#1})}} % One single link

%\linespread{1.2}%
%\selectfont

\begin{document}

\LRCornerWallPaper{1}{bg.jpg}

\maketitle\nocite{raffael}
\thispagestyle{empty}

\newpage
\ClearWallPaper

\tableofcontents

\newpage

\thispagestyle{empty}
\centeredquote{Jede gute Software wird von einem Entwickler geschrieben, der ein persönliches Problem lösen will.}{Eric S. Raymond\footfullcite{kathedralewiki}}

\newpage

\section{Typographische Konventionen}
\label{typographie}

\begin{technisches}
Sehr technische oder abstrakte Erklärungen, die nicht gelesen werden müssen, um die Software
normal zu benutzen, werden mit dem Symbol \frq \mydbend\flq\ gekennzeichnet. Außerdem sind solche
Bereiche grau hinterlegt, um sie optisch hervorzuheben.
\end{technisches}

Verweise auf andere Sektionen werden mit einem \ExternalLink-Zeichen gekennzeichnet, kursiv geschrieben
und mit der Sektionsnummer versehen. In der PDF sind sie außerdem anklickbar und verlinken direkt auf
den referenzierten Verweis. Beispiel: \fullref{typographie}.

Die gesamte Dokumentation wurde mit \XeLaTeX\ in der Schriftart \frq Noto\flq\ gesetzt.

\section{Einleitung}

Die Vorlesungsverzeichnissoftware \frq VVZ\flq\ (von \textbf{V}orlesungs\textbf{v}er\textbf{z}eichnis)
versammelt an einem zentralen Ort alle
für die Planung des Studiums notwendigen Daten. Mit ihm ist es möglich, herauszufinden,
wann wo welche Veranstaltung stattfindet, sich einen Stundenplan zu erstellen und die 
Termine automatisch in den digitalen Kalender einzutragen.

\section{Zu den Cookies}

\begin{technisches}
Cookies sind kleine Textdateien, die auf Ihrem Rechner gespeichert und beim Aufrufen des VVZs an den
VVZ-Server gesendet werden. In diesen sind dann Informationen wie ausgewählte Prüfungsleistungen, Anmeldedaten
für den Administrationsbereich oder ausgewählte Veranstaltungen befinden. Das VVZ nutzt diese Daten nicht, um
Sie zu verfolgen und diese werden auch nirgendwo sonst preisgegeben oder von der Serverseite gespeichert.
Lediglich beim Aufrufen der Seite sind die Cookies im Arbeitsspeicher des Servers, um die Anfrage
bearbeiten zu können.

Die Seite ist auch ohne Cookies weitgehend benutzbar, aber Funktionen wie das Anmelden im Administrationsbereich,
Speichern von Veranstaltungen oder Prüfungsleistungen benötigen Cookies zum Funktionieren.
\end{technisches}




%\section{Ist es nicht irgendwie gefährlich, die interne Struktur zu veröffentlichen?}
%
%\begin{technisches}
%Nein, denn Sicherheit basiert nicht auf Unwissenheit (\frq Security through Obscurity\flq), sondern 
%auf Offenheit. Es gilt Kerckhoffs' Prinzip: \frq [Ein sicheres
%System] darf nicht der Geheimhaltung bedürfen und soll
%ohne Schaden in Feindeshand fallen können\flq\ (Quelle: \cite{kerkhoffwiki}). Dies gilt für alle
%sichere Software. Security-through-Obscurity bietet immer nur Scheinsicherheit.
%\end{technisches}

\section{Benutzung und Details}

\subsection{Anmelden}

Das Anmelden findet (noch) nicht über Shibboleth, sondern über ein eigenes Anmeldesystem statt. Accounts
zum Anmelden können Ihnen Herr Dr. Holm Bräuer, Herr Norbert Engemaier oder ich (die Administratoren der Seiten)
erstellen.

In der Logikmaske können Sie Ihren Namen (ohne Titel) und Ihr Passwort eingeben und sehen die Startseite.

\begin{technisches}
Es gibt mehrere Arten von Accounts:

\begin{itemize}
	\item Administratoren (haben Zugriff auf alle Seiten und können alles ändern)
	\item Dozenten (haben Zugriff auf die eigenen Daten und können eigene Veranstaltungen bearbeiten)
	\item Superdozenten (haben, wie Dozenten, Zugriff auf eigene Veranstaltungen, aber können auch die
		Veranstaltungen von ausgewählten anderen \frq superverwalten\flq, d.\,h. anlegen, bearbeiten
		und löschen)
	\item Verwalter (haben Zugriff auf Verwaltungsfunktionen, können aber keine Veranstaltungen erstellen,
		bearbeiten oder löschen)
	\item Raumplanungsdozenten (können ihre eigenen Veranstaltungen erstellen, bearbeiten und löschen und
		haben Zugriff auf die Raumplanungsseite, vgl. \fullref{raumplanung})
\end{itemize}

	Jeder Account ist exakt einem dieser sogenannten \frq Rollen\flq\ zugeordnet (vgl. \fullref{rollen}).

\end{technisches}

\subsection{Abmelden}

Nach der Anmeldung sind sowohl im Back- als auch im Frontend oben auf der Seite Links, mit denen Sie sich abmelden
können.

\subsection{Vorteile der eingeloggten Nutzer im Frontend}

Wenn Sie Rechte haben, Veranstaltungen zu bearbeiten, sehen Sie im Frontend unter jeder Veranstaltung, die Sie
bearbeiten dürfen, einen Link: \frq diese Veranstaltung bearbeiten\flq. Somit können Sie direkt von der Startseite
zur \frq Einzelne Veranstaltung\flq-Seite Ihrer Veranstaltung.

\begin{technisches}
Wenn Sie Administrator sind, haben Sie mehr Rechte. So sehen Sie z.\,B. die Debug-Informationen am unteren Ende der
Seite.
\end{technisches}

\subsection{Der Aufbau der Seite}

Die Seite ist vier- (bzw. fünf)-geteilt. Im obersten Abschnitt finden Sie nach der Anmeldung Ihren Namen, das
TU-Dresden-Logo und einen Link zum Abmelden Ihres Accounts.

Darunter befindet sich eine Zeile mit allen Unterseiten und Seitenkategorien. Unterseiten erkennen Sie an der
dunkleren Farbe, Seitenkategorien (thematische Sammlung einzelner Unterseiten) erkennen Sie am Pfeil runter $\downarrow$ 
neben dem Namen und dem gräulicherem Blauton. Die aktuell ausgewählte Seite hat sowohl einen Pfeil $\longrightarrow$, als
auch eine deutliche grüne Hintergrundfarbe.

Darunter befindet sich der Hauptbereich für einzelne Unterseiten. Dort werden -- je nach ausgewählter Unterseite --
die Inhalte der Seite angezeigt.

Noch weiter runter befindet sich --- wie im Frontend --- die Leiste mit nützlichen Links auf der Seite.

\begin{technisches}
Darunter befindet sich der Query-Debugger, der nur für Administratoren sichtbar ist. In ihm sind alle Queries, die für den
Aufbau der Seite ausgeführt worden sind, Debug-Meldungen, Stack-Traces usw.
\end{technisches}

\begin{technisches}
Alle Daten auf der Seite sind \frq normalisiert\flq, das heißt: eine Änderung an einer Stelle wirkt sich überall aus.
Als Beispiel sei ein Dozent genannt, der durch Heirat seinen Namen ändert. Der neue Name muss nur an einer Stelle
einmalig eingetragen werden und ist dann überall sofort aktualisiert. Es gibt keine redundanten Daten, die an mehreren
Stellen geändert werden müssen.
\end{technisches}

\begin{technisches}
Vor jedem Löschen einer Information aus dem VVZ werden alle Abhängigkeiten, die durch die Normalisierung entstehen,
gesucht und angezeigt. Beispiel:

Der Datenbankzustand muss immer konsistent sein, damit immer alle nötigen Informationen vorhanden sind. Löscht man
aber nun einen Dozenten, dann hätten Veranstaltungen ohne seine Existenz im System keine Zuordnung mehr.
Daher müsste man alle Veranstaltungen mitlöschen. 

Um klar anzuzeigen, was entfernt werden würde, wenn man Etwas löscht, wird ebendies alles angezeigt und man kann
noch von der Entscheidung, das zu löschen, zurücktreten.

Bei vielen Menüpunkten besteht daher auch die Möglichkeit, einen Eintrag zu deaktivieren statt ihn zu löschen.
Das Deaktivieren verhindert z.\,B. bei Accounts, dass man sich mit dem Benutzeraccount anmelden kann, ohne,
dass die Daten, die ihm zugehörig sind, gelöscht werden müssen. Solche Mechanismen existieren bei den 
\fullref{accounts}, bei den \fullref{pruefungsnummern} und bei \fullref{dozenten}.
\end{technisches}

In den meisten Tabellen, in denen Daten stehen, steht ganz rechts der Button \frq löschen\flq. Damit lassen sich, für
Accounts, die die jeweiligen Rechte besitzen, Daten aus dem VVZ löschen. Die letzte Spalte der meisten Tabellen 
beinhaltet dagegen die Möglichkeit, einen neuen Eintrag hinzuzufügen. Dieser ist dann leer und hat keinen
\frq Löschen\flq-Button.

\section{Ich will doch nur\dots}

\subsection{\dots\ meine Veranstaltung eintragen, ändern oder löschen}

Dazu müssen Sie einen Account mit den Rechten, eine Veranstaltung zu erstellen, besitzen und angemeldet sein.
Gehen Sie in der Navigationsleiste auf \frq Veranstaltungen\flq, scrollen runter und tragen in die letzte Zeile
der großen Tabelle den Namen, das Semester und den Typ ein (und, wenn Sie Administrator oder Superdozent sind,
den Namen des Dozenten; sonst wird automatisch der mit Ihrem Account verknüpfte Dozent eingetragen).

Klicken Sie dann auf \frq Bearbeiten\flq. Das speichert die eingegebenen Daten und leitet Sie weiter zur
Seite \frq Einzelne Veranstaltungen\flq\ (siehe \fullref{einzelneveranstaltung}). Dort können Sie Zeiten, Räume und
Prüfungsleistungen eintragen. Klicken Sie dann entweder ganz oben oder ganz unten auf der Seite nochmal auf
\frq Speichern\flq\ und die Veranstaltung ist sofort live. 

Über den selben Weg können Sie die Veranstaltung auch nachträglich jederzeit bearbeiten.

Der Gebäude- und Raumwunsch wird nicht automatisch veröffentlicht. Dazu muss erst der Raumplanungsdozent
nach der Kommunikation mit dem Raumplanungsdezernat eintragen, wo die Veranstaltung stattfindet.

\subsection{\dots\ mein Passwort ändern}

Das können Sie auf der Seite \frq Eigene Daten ändern\flq\ in der Navigationsleiste.

\subsection{\dots\ mein vergessenes Passwort erfahren}

Die Passworte werden nicht im Klartext gespeichert, sondern gehasht. Das heißt, dass niemand Zugriff auf Ihr
verlorenes Passwort hat. Aber die Administratoren können Ihr Passwort jederzeit neu setzen. Benutzen Sie einfach
die Kontakt-Seite am unteren Bildschirmrand, um uns danach zu fragen.

\begin{technisches}
Genauer gesagt sind die Passworte mit \texttt{sha256} und einem zufällig-generiertem Salt gehasht. Es ist 
nicht möglich, aus dieser Kombination Ihr Ursprungspasswort wiederherzustellen. \texttt{sha256} ist eine
sogenannte Einwegfunktion. Das heißt, dass es relativ einfach ist, $y$ mit $y = f(x)$ zu berechnen, aber praktisch
unmöglich, aus $y$ herauszufinden, welches $x$ (Ihr Passwort) in die Funktion gegeben worden ist. Das Hinzufügen
eines Saltes erschwert das noch zusätzlich, weil die zufällig angehangene Zeichenkette sogenannte 
\frq Rainbow-Tables\flq\ in der Praxis 
unmöglich macht, in der eine Menge aus verschiedenen $x \leftrightarrow y$-Kombinationen gespeichert sind.
\end{technisches}

\section{Unterseiten}

\subsection{Willkommen!}

Auf der Startseite sehen Sie Informationen und Hinweise zur Benutzung. Darunter sehen Sie die gleichen Daten, wie sie
oben im Menü angezeigt werden, in hierarchischer Struktur. Von hier aus können Sie alle Seiten, die Sie aufrufen dürfen,
auf einen Blick sehen.

\begin{technisches}
Sowohl die Leiste oben als auch diese Ansicht passt sich an die Rolle des angemeldeten Benutzers an. Ein Administrator
sieht bei Beidem andere Punkte als ein Raumplanungsdozent.
\end{technisches}

\subsection{Eigene Daten ändern}

Auf \frq Eigene Daten ändern\flq\ können Sie Ihr eigenes Passwort ändern und festsetzen, ob der Raumplanungsdozent
automatisch sehen soll, dass Sie ein barrierefreies Gebäude bevorzugen. Mit dieser Option müssen Sie das nicht immer
manuell eintragen beim Erstellen von Veranstaltungen.

\subsection{Kontakt}

Unter Kontakt können Sie -- wie unten in der Navigationsleiste -- zum Kontaktformular kommen, um Fehler, Vorschläge und
Fragen zur Software oder den Daten zu geben. Als angemeldeter Benutzer muss man keine Sicherheitsfrage eingeben.

\subsection{Moduldaten}

In dieser Kategorie befinden sich zwei Unterseiten, die dazu dienen, Informationen zu Modulen einzutragen oder zu ändern.

\subsubsection{Modul $\longrightarrow$ Semester}

Hier können Informationen eingetragen werden, die für die automatische Stundenplanerstellung genutzt werden. Dabei muss
man als erstes einen Studiengang auswählen und bekommt dann eine Liste aller Module. Zu diesen Modulen kann man dann sagen,
in welchem Semester sie laut Regelstudienordnung auftauchen sollten. Dies geht mit einem einfachen Komma:
\frq\texttt{1,2}\flq\ heißt: im ersten und zweiten Semester.

Sind die Semesterinformationen gespeichert, taucht eine zweite Tabelle auf, in der die Anzahl der Veranstaltungstypen, der
Credit-Points und Prüfungsleistungen, die nötig sind, eingetragen werden können.

Ist dies für einen Studiengang eingetragen ist eine automatische Stundenplanerstellung möglich.

\subsubsection{Module}

Auf der Unterseite zu Modulen können Sie neue Module erstellen, vorhandene bearbeiten oder löschen. Dabei braucht
jedes Modul eine Zuordnung zu einem Studiengang, einen Namen und optionalerweise auch eine Beschreibung und eine Abkürzung.

Studiengänge hierfür können unter Stammdaten $\longrightarrow$ Studiengänge hinzugefügt werden (vgl. \fullref{studiengaenge}).

\subsection{Prüfungsdaten}

In dieser Kategorie sind alle Unterseiten, die Etwas mit der Verwaltung von Prüfungsleistungen zu tun haben.

\subsubsection{Dozent $\longrightarrow$ Prüfungsdaten}

\label{dozentpruefungsdaten}

Diese Seite erstellt automatisch Berichte für die Prüfungsämter, welcher Dozent welche Prüfungsleistungen angeboten hat.
Dabei kann man sowohl Semester, Studiengang als auch das Prüfungsamt einschränken (manche Prüfungsämter wollen nicht
die Daten aller Studiengänge, und mit der Definition von Prüfungsämtern kann man festlegen, welche was bekommen wollen, vgl. \fullref{pruefungsaemter}).

Am unteren Ende der Seite besteht die Möglichkeit, die oben angezeigte Liste als Excel-Datei herunterzuladen.

Alles auf dieser Seite wird vollautomatisch erstellt. Es ist nicht möglich, hier etwas zu ändern (dazu müssten 
beispielsweise Dozenten in ihren Veranstaltungen Prüfungsleistungen abwählen, siehe \fullref{einzelneveranstaltung}).

\subsubsection{Prüfung $\longrightarrow$ Zeitraum}

Jeder Prüfungsleistung ist ein Zeitraum zugeordnet (\frq erster Zeitraum\flq\ und \frq zweiter Zeitraum\flq). Diese
Zeiträume lassen sich hier erstellen, ändern oder löschen.

\subsubsection{Prüfungsämter}

\label{pruefungsaemter}
Für die in \fullref{dozentpruefungsdaten} auswählbaren Prüfungsämter lässt sich hier festlegen, welche Prüfungsämter an
Dokumenten welcher Studiengänge interessiert sind. In der oberen Tabelle lassen sich neue Prüfungsämter eintragen oder alte
umbenennen oder löschen und in der unteren Tabelle lassen sich für jedes Prüfungsamt Checkboxen setzen für gewisse
Studiengänge.

\subsubsection{Prüfungsnummer $\longrightarrow$ Dozent}

Ähnlich wie in \fullref{dozentpruefungsdaten} lässt sich hier vollautomatisch erstellt eine Liste ansehen und als Excel-Datei
downloaden, nur ist hier Zuordnung hier genau andersherum. Jeder Prüfungsleistung wird zugeordnet, welche Dozenten sie
in dem Semester anbieten und in welchem Studiengang.

\subsubsection{Prüfungsnummern}

\label{pruefungsnummern}

Die Prüfungsnummernunterseite erlaubt es, Prüfungsnummern zu erstellen, zu bearbeiten oder zu löschen. Jeder 
Prüfungsleistungen ist dabei ein Studiengang, ein Modul, ein Bereich, ein Prüfungstyp, eine Modulbezeichnung (optional),
eine Prüfungsnummer (optional) und ein Zeitraum zugeordnet. Einzelne Prüfungsleistungen lassen sich hier auch
deaktivieren, um die Datenkonsistenz nicht zu gefährden und trotzdem Veranstaltungen auf der Seite einzelner 
Veranstaltungen nicht sichtbar werden zu lassen (vgl. \fullref{einzelneveranstaltung}).

\subsubsection{Prüfungstypen}

Diese Unterseite verwaltet die Arten von Prüfungsleistungen, die es gibt, beispielsweise Klausuren oder Essays.
Bis auf den Namen kann man hier keine Informationen eintragen.

\subsection{Raumplanung}

\label{raumplanung}

Auf der Raumplanungsseite ist sowohl das Institut als auch das Semester auswählbar. Hier erscheinen alle für die
gewählten Einstellung eingetragenen Veranstaltungen mit Zusatzinformationen wie dem Gebäudewunsch und dem Raumwunsch,
aber auch einem vom Dozenten frei eingebbaren Hinweis für die Raumplanung.

Der Raumplanungsdozent kann hier -- nach Rücksprache mit dem Raumplanungsdezernat -- ein Gebäude und einen Raum für eine
Veranstaltung festlegen. Diese erscheinen dann im Frontend des VVZs. Die Raumwünsche und Gebäudewünsche erscheinen
dort noch nicht. Zusätzlich kann eingetragen werden, wann der Raum gemeldet worden ist.

Wenn der Raumplaner einen Raum eingetragen hat, aber der Dozent noch für die Raumplanung relevante Daten geändert hat,
erscheint neben dem Titel der Veranstaltung ein kleiner Blitz {\Huge\textcolor{yellow}{\faBolt}}. Dieser weist darauf hin, dass nachträglich
relevante Daten geändert worden sind und man sich die Veranstaltung nochmal ansehen sollte. Durch erneutes Speichern (auch,
wenn keine weiteren Änderungen vorgenommen worden sind) verschwindet der Blitz.

Am unteren Ende der Seite ist die Möglichkeit, sich die oben gezeigte Liste als Excel-Datei zu downloaden.

\subsection{Stammdaten}

\subsubsection{Alte Daten löschen}

Wenn Daten, die vergangene Semester betreffen, aus irgendwelchen Gründen gelöscht werden sollen, kann man das hier machen.
Die Seite zeigt alle Semester, die die Software kennt, an, und gibt die einfache Möglichkeit zum konsistenten Löschen.

\subsubsection{Bereiche}

Hier können Bereichsnamen erstellt, bearbeitet oder gelöscht werden.

\subsubsection{Dozenten}

\label{dozenten}

Hier können Dozenten mit ihren Metadaten erstellt, bearbeitet oder gelöscht werden. Auch kann hier festgelegt
werden, ob der Dozent noch am Institut oder bereits ausgeschieden ist. In der Tabelle darunter ist eine Übersicht,
wie viele Prüfungsleistungen ein Dozent im ausgewählten Semester anbietet.

Noch etwas weiter runter ist eine Tabelle, in der eine Verknüpfung zwischen Dozent und Titel hergestellt werden kann.
Hat Jemand z.\,B. einen Doktortitel und man trägt das hier ein, taucht der Doktortitel auf der Seite an den meisten
Stellen auf, wo der Name des Dozenten steht (für die Festlegung der Titel, siehe \fullref{titel}).

\subsubsection{Einzelne Veranstaltung}

\label{einzelneveranstaltung}
Auf der Unterseite \frq Einzelne Veranstaltung\flq\ können --- wie der Name es sagt --- einzelne Veranstaltungen
bearbeitet werden. Geht man über das Menü auf diese Seite, erscheint eine Liste aller Veranstaltung, die Sie editieren
dürfen. Wählen Sie dort eine aus und klicken auf \frq Veranstaltung auswählen\flq, und Sie können die Veranstaltung
bearbeiten.

In der ersten Tabelle haben Sie die Möglichkeit, Sprache, Wochentag, Stunde, Wochenrhythmus,
den ersten Termin, die Anzahl der Hörer (für die Raumplanung), einen Link zu OPAL, einen Gebäude- und einen Raumwunsch
einzutragen. Dabei ist alles -- bis auf Wochentag, Stunde und Wochenrhythmus -- optional. Umso mehr Informationen Sie
aber eintragen, umso nützlicher ist die Seite.

Darunter ist eine Tabelle mit zwei Hinweisfeldern. Der Hinweis für die Raumplanung taucht nur auf der Raumplanungsseite
auf (vgl. \fullref{raumplanung}), der Hinweis für die Studenten auch im Frontend.

Darunter haben Sie die Möglichkeit, eine inhaltliche Verbindung zweier Fächer herzustellen, beispielsweise einer Vorlesung
und einer Übung. Diese Verbindung erscheint dann auf der Startseite.

Dann kommt eine Tabelle, in der einzelne Termine und Orte eingetragen werden können, die nicht in das oben-angegebene
Muster passen (ist die ganze Veranstaltung nicht ins Schema zu pressen, kann man dafür bei der Stunde \frq *\flq, oder
\frq ganztägig\flq\ auswählen und beim Tag \frq BS\flq\, für Blockseminar, auswählen).

Darunter befindet sich eine Tabelle aller Prüfungsleistungen. Diese sind geordnet nach Studiengängen (blaue Überschriften)
und Modulen (linke Seite). Wird hier ein Haken gesetzt, dann taucht die Veranstaltung in der Unterseite des Studienganges
des Modules auf, für das eine Prüfung ausgewählt worden ist. Tipp: mit der Tastenkombination \frq STRG f\flq\ kann man auf
einer Internetseite z.\,B. nach Zahlen oder Text suchen. Somit finden sich die Prüfungsleistungen schneller. 

Darunter ist ein großer Speichernknopf, der auch oben auf der Seite nochmal ist. Klickt man diesen, werden alle 
eingetragenen Informationen gespeichert.

Noch weiter runter ist die Möglichkeit, ausgewählte Prüfungsleistungen von einer anderen Veranstaltung zu klonen (z.\,B.
aus einem vorherigen Semester). Dabei kann man sich Veranstaltungen alter Semester anzeigen, aber auch aktueller, auf
die man selbst Zugriff hat. Außerdem lassen sich mit einer Checkbox die aktuell ausgewählten und die zu-klonenden 
in einer Gesamtmenge vereinen, so, dass man beide hat.

\subsubsection{FAQ}

Hier ist es möglich, die Daten des FAQs zu editieren. Dazu können Fragen und Antworten eingegeben werden, und eine
Zahl, wie häufig danach gefragt wird. Umso häufiger etwas gefragt wird, umso höher steht es automatisch im FAQ.

\subsubsection{Gebäude}

Hier lassen sich Gebäude einfügen, bearbeiten oder löschen. Dabei muss man den Gebäuden sowohl einen Namen als auch eine
Abkürzung geben.

\subsubsection{Institute}

Auf der Seite der Institute lassen sich neue Institute erstellen, vorhandene bearbeiten oder löschen. Dazu hat jedes 
Institut einen Namen und eine Startnummer. Diese Startnummer ist relevant für die Liste auf der Raumplanungsseite,
denn für die Prüfungsämterlisten beginnen Veranstaltungen eines Institutes immer mit einer bestimmten, automatisch
generierten ID. Die Startnummer wird der automatisch generierten \frq 1\flq\ für die erste Veranstaltung aufaddiert.
Ist sie nicht gesetzt, wird das behandelt wie eine 0.

\subsubsection{Merges}

\begin{technisches}
Die Merge-Seite erlaubt es, sehr ähnliche Informationen in der Datenbank zu verschmelzen. Als Beispiel sei 
genannt, dass man ausversehen zwei Sprachen eingetragen hat, die eigentlich eine sind, und beiden Sprachen
bereits mit Veranstaltungen verknüpft sind. Würde man eine davon löschen, wäre die Datenbank inkonsistent. 

Merges erlaubt es, solche Strukturen automatisch zusammenzufügen, in dem man eine geeignete Tabelle auswählt
(der automatische Vorgang geht nicht mit allen Tabellen!) und in der oberen Tabelle eine Liste aller Einträge hat
mit Checkboxen und darunter mit Radio-Buttons.

Wählt man in der ersten Tabelle drei Einträge aus und in der zweiten nur einen, dann werden alle Referenzen der
auf Daten in der Tabelle geändert auf das, was in der zweiten Tabelle ausgewählt ist. Danach werden die alten,
nicht-mehr-referenzierten Daten gelöscht.

\mydbend\ \mydbend\ \mydbend\ Achtung: das ist ein sehr experimentelles Feature. Ich habe bisher bei vielen
Versuchen noch keine Fehler entdecken können, aber ich würde dringend empfehlen, vorher ein Datenbankbackup
zu machen, siehe dazu \fullref{backup}.
\end{technisches}

\subsubsection{Semester}

Im Unterpunkt Semester sind die vom System automatisch erstellten Jahressemester (Sommersemester/Wintersemester und Jahr).
Das System erstellt automatisch Semester für die nächsten zehn Jahre, um immer die Möglichkeit zu bieten, genügend
Referenzpunkte für Veranstaltungen zu haben.

Hier kann außerdem das Start-Datum des Semesters eingetragen werden. Einmal erstellte Semester können nicht mehr über
die grafische Oberfläche entfernt werden, so wie neue Semester auch nicht eingetragen. Das geschieht immer automatisch.

\subsubsection{Sprachen}

Hier können Sprachen festgelegt werden, die im Menüpunkt \frq Einzelne Veranstaltung\flq\ ausgewählt werden können,
     siehe dazu \fullref{einzelneveranstaltung}. Dazu muss je ein Name und eine Abkürzung (bestenfalls nach ISO 639-5) festgelegt 
werden.

\subsubsection{Studiengänge}

\label{studiengaenge}

Hier können neue Studiengänge eingeführt, vorhandene bearbeitet oder gelöscht werden. Dazu muss je ein Institut angeben
werden, ein Name und optionalerweise auch ein Link zur Studienordnung. Dieser erscheint dann im Frontend.

\subsubsection{Stundenpläne}

Unter \frq Stundenpläne\flq\ sind Links zur automatischen Stundenplanverwaltung, die für jeden Dozenten (bei 
Administratoren) bzw. für sich selbst den Stundenplan enthält. Dieser wird automatisch generiert aus den ausgewählten
Veranstaltungen.

\subsubsection{Superdozenten}
\label{superdozenten}

Diese Seite erlaubt das Festlegen der Superdozenten. Ein Superdozent ist ein Dozent, der Superkräfte hat. Das heißt:
er kann Veranstaltungen ausgewählter anderer Dozenten bearbeiten. Dazu muss einem Dozenten die Rolle als Superdozent zugewiesen werden (siehe \fullref{rollen}) und Checkboxen bei den Namen der jeweiligen Dozenten gemacht werden.

\subsubsection{Titel}
\label{titel}

In der \frq Titel\flq-Seite können zentral Titel festgelegt werden, die Dozenten zugewiesen werden können (vgl. auch
\fullref{dozenten}). Dabei hat jeder Titel einen Namen und eine Abkürzung, wobei nur die Abkürzung vor den Namen
erscheint.

\subsubsection{Vorlesungstypen}

Hier können Vorlesungstypen (beispielsweise Vorlesung oder Textproseminar) eingepflegt werden. Dabei müssen sowohl 
ein Name als auch eine Abkürzung eingegeben werden.

\subsection{System}

\subsubsection{Accounts}
\label{accounts}

Hier können Accounts festgelegt werden, mit denen es möglich ist, sich im Backend anzumelden und beispielsweise
Veranstaltungen zu erstellen. Dabei muss ein Name, ein Passwort, eine Rolle und optionalerweise ein zugeordneter
Dozent eingetragen werden. Ist ein Dozent zugeordnet, kann ein Account der Dozentenklasse (Dozent oder Raumplanungsdozent)
nur die Veranstaltungen dieses Dozenten bearbeiten. Ausnahme ist der Superdozent, der auch Veranstaltungen ausgewählter
anderer Dozenten bearbeiten darf. Hier lässt sich auch festlegen, ob automatisch in der Raumplanung ein Hinweis auftauchen
soll, dass barrierefreie Räume bevorzugt werden.

Dahinter ist auch eine Checkbox, ob die Datenschutzfrage akzeptiert worden ist. Diese lässt sich nicht beim Erstellen
eines Benutzers setzen, um sicherzustellen, dass jeder zugesagt hat, der die Software benutzt. Sollte je eine neue
Datenschutzfrage nötig sein (z.\,B. wegen einer Gesetzesänderung), ist es hier jederzeit durch Wegmachen des Hakens
möglich, den Benutzer zum erneuten Akzeptieren (oder Ablehnen) zu bringen.

Außerdem lassen sich hier Accounts deaktivieren, um sie nicht löschen zu müssen und Dateninkonsistenz zu verursachen.

\begin{technisches}
Accounts sind nicht das Gleiche wie Dozenten. Es kann Accounts ohne Dozenten geben und umgekehrt. Dozenten ohne Accounts
können beispielsweise von Superdozenten oder Superduperdozenten (umgangssprachlich Administratoren) 
verwaltet werden. Vgl. dazu \fullref{superdozenten}.
\end{technisches}

\begin{technisches}
Die Passworte sind nicht im Klartext in der Datenbank, sondern \texttt{sha256}-gehasht mit einem langen,
zufällig-generiertem Salt. Daher ist es nicht möglich, vergessene Passworte auszulesen. Wenn ein Nutzer sein Passwort
vergisst, können die Administratoren hier jedoch ein neues Passwort setzen.
\end{technisches}

\subsubsection{Aktuelles Semester setzen}

Hier ist es möglich, festzulegen, welches als Semester als das Aktuelle anzusehen ist. Dieses wird überall, wo eine
Semesterauswahl möglich ist, als Standardsemester gewählt.

\subsubsection{Apache-Neustarts}

\begin{technisches}
Einmal die Minute überprüft ein Cronjob, ob der Apache noch läuft und die Seite erreichbar ist. Ist das nicht der Fall,
wird versucht, den Webserver neuzustarten. Wenn das auftritt, wird das in die Datenbank geloggt und auf dieser Seite
angezeigt.
\end{technisches}

\subsubsection{API}

\begin{technisches}
Die API erlaubt es, von Außen Informationen aus dem VVZ abzurufen, z.\,B., um eigene Apps zu entwickeln.
Siehe dazu die Frontend-Dokumentation.

Auf dieser Seite ist es möglich, neue Nutzer einzutragen und für sie einen Auth-Code zu erstellen, mit dem der Zugriff auf
die API gewährt wird. Dazu kann man einen Namen, eine Kontakt-Email-Adresse und eine Begründung eintragen, wofür die API
benutzt werden soll. Dies soll ermöglichen, wenn plötzlich sehr viele Anfragen stattfinden, nach dem Rechten zu fragen,
um Softwarefehler o.\,Ä. in der Software Anderer zu finden, die diese API benutzen und eventuell den VVZ-Server durch
zu viele Anfragen belasten.

Die API-Zugriffe werden mitgeloggt, um potenzielle Botnetze erkennen zu können. Dafür kann man sich auf dieser Seite auch 
die Logs anzeigen.
\end{technisches}


\subsubsection{DB-Backup}

\label{backup}
\begin{technisches}
Hier ist für Administratoren es möglich, sich einen Dump der Datenbank zu holen, um sie zu sichern. Auch ist hierüber
das komplette Wiederherstellen eines vorherigen Datenbankzustandes durch einen gespeicherten Dump möglich. Die Änderungen
zwischen Dump und aktuellem Status lassen sich im DB-Diff nachsehen, siehe dazu \fullref{dbdiff}.
\end{technisches}

\subsubsection{DB-Diff}

\label{dbdiff}
\begin{technisches}
Hier kann eine MySQL-Dump-Datei hochgeladen werden und es erscheint eine Liste der Änderungen zwischen Dump und aktueller
Datenbankversion.
\end{technisches}

\subsubsection{Funktionen}
\label{funktionen}

\begin{technisches}
Das VVZ verhindert unauthorisierte Änderungen über das Rollensystem, das jedem Account eine Rolle zuordnet. Vor jeder
Funktion, die potenziell gefährlich sein könnte, weil damit Änderungen in der Datenbank gemacht werden könnten, wird
überprüft, ob der aktuell-eingeloggte Account die Rechte besitzt, diese Änderungen zu machen. Dafür ist hier eine Liste
aller Funktionen vorhanden (obere Tabelle), wo neue Funktionen angelegt werden können.

In der unteren Tabelle ist eine Liste aller Funktionen und eine Liste von Rollen mit Checkboxen. Wählt man für eine Funktion
eine Checkbox aus, darf der Benutzer sie ausführen. Sonst erhält er eine Fehlermeldung. Zur Rechteverwaltung, siehe Rollen
unter \fullref{rollen} und Neue Seite unter \fullref{neueseite}.
\end{technisches}

\subsubsection{Neue Seite}

\label{neueseite}

\begin{technisches}
Der Menüpunkt \frq Neue Seite\flq\ erlaubt es, neue Seiten in die Navigation hinzuzufügen. Dabei muss ein Name (der
überall erscheint, wo die Seite verlinkt wird), eine \texttt{.php}-Datei, die im Ordner \texttt{./pages/} liegen muss,
eine Checkbox, ob die Seite in der Navigation angezeigt werden soll, ein Vaterelement (was die Seite zu einer Kategorie
in der Navigation zuordnet, Rollen, die Zugriff haben und eine Beschreibung (die in den Kategorien und auf der 
Willkommensseite angezeigt wird) und ein Hinweis, der auf der Seite selbst angezeigt werden, festgelegt werden.

Über diesen Menüpunkt lassen sich auch die bisherigen Zuordnungen verändern, wenn man z.\,B. den Punkt 
\frq Raumplanung\flq\ in die Kategorie der Stammdaten verschieben will, dann kann man das hier ändern und speichern.
\end{technisches}

\subsubsection{Rollen}

\label{rollen}

\begin{technisches}
In der Unterseite der Rollen ist es möglich, vorhandene Rollen zu bearbeiten, zu löschen oder neue anzulegen. Eine Rolle
bestimmt, was einem Nutzer angezeigt wird und was er verändern darf (siehe auch dazu \fullref{funktionen}). Damit eine neue
Rolle eine neue Seite vollständig benutzen darf, müssen alle auf der Seite benutzten Funktionen eingetragen und für die 
Rolle freigeschaltet sein und die Rolle braucht auch Zugriff auf die Seite selbst. 

Zusätzlich zur Seitenzuordnung kann jeder Rolle auch eine kurze Beschreibung gegeben werden.
\end{technisches}

\subsubsection{Seiteninformationen}

\begin{technisches}
Beinhaltet nur die Untermenge der Daten aus \frq Neue Seite\flq\ (\fullref{neueseite}), bestehend aus der Beschreibung und
dem Hinweis für eine Seite. Diese Unterseite ist damit obsolet. Aber sie greift auf die gleichen Daten zu, wie die
\frq Neue Seite\flq-Seite und daher ist es egal, auf welcher von Beiden Hinweise oder Beschreibungen geändert werden.
\end{technisches}


\subsubsection{User-Agents}

\begin{technisches}
Beinhaltet Listen, welche Browser, Betriebssysteme und Browser-Betriebssystem-Kombinationen auf die
Seite zugegriffen haben und wie oft. Bietet auch die Möglichkeit, sich verschiedene Bot-Zugriffe anzeigen zu lassen,
z.\,B. vom Googlebot.
\end{technisches}

\subsection{Veranstaltungen}

\label{veranstaltungen}

Zeigt alle Veranstaltungen, die der eingeloggte Benutzer bearbeiten darf und erlaubt es auch, Veranstaltungen aus
vorherigen Semestern (bzw. bei Administratoren und Superdozenten von einzelnen Dozenten anzuzeigen, auf die man
Zugriff hat).

Dabei hat jede Veranstaltung einen Namen, ein zugeordnetes Semester, einen zugeordneten Dozenten und einen zugeordneten
Veranstaltungstyp. Außerdem wird angezeigt, wie viele Prüfungsleistungen in der Veranstaltung ausgewählt sind. Ist die
Anzahl 0, wird sie rot und mit drei Ausführungzeichen davor und danach markiert.

Will man nur die Daten der Veranstaltung ändern, die auf dieser Seite angezeigt werden, kann man auf
\frq Speichern\flq\ klicken. Will man die Einzelne-Veranstaltungs-Seite (vgl. \fullref{einzelneveranstaltung}) dieser
Veranstaltung haben, um Zeit, Raum, ausgewählte Prüfungsleistungen usw. zu ändern, muss man auf
\frq Bearbeiten\flq\ klicken; dabei wird auch die Änderung auf der aktuellen Seite gespeichert, aber an gelangt an die
spezifische Seite der einzelnen Veranstaltung).

In der letzten Zeile lassen sich neue Veranstaltungen eintragen und statt der Anzahl der Veranstaltungen, die
ausgewählt wird, steht dort die Summe der oben ausgewählten Veranstaltungen (mit $\uparrow\Sigma = \dots$).

\section{Zukünftige Ideen}

\begin{itemize}
	\item Zentrales Eingeben von Verspätungen, Verlegungen und Ausfällen mit intuitiven Auswahlmöglichkeiten
		(\frq nächste Woche\flq, \frq die nächsten $n$ Wochen\flq,
		\frq alle zwei Wochen bis \texttt{DATUM}\flq, \dots). Daraus resultierend:
		\begin{itemize}
			\item Zentraler Vertretungsplan
			\item Automatische Statistik über Vertretungspläne im einheitlichen Format
		\end{itemize}
	\item Zentrale Quellen- und Literaturverwaltung. Daraus resultierend:
		\begin{itemize}
			\item Eine Liste aller verwendeten Bücher, Skripte und Websites (mit Titel, ISBN,
				Veröffentlichungszeitraum, \dots)
			\item Einfache Eingabe von Literatur (z.\,B. Vorschläge von Büchern, die bereits eingetragen 
				worden sind)
			\item Automatische Statistiken über Lesestoffe und Quellen
		\end{itemize}
\end{itemize}

\section{Fehlerbehebung}

Bei gravierenden Fehlern wird automatisch eine Fehlerseite angezeigt und die Administration informiert. 

Kleinere Probleme lassen sich meist selbst beheben. Wenn das Aufklappen von Menüs nicht funktioniert,
aktivieren Sie bitte Javascript. Für die Anmeldung auf der Administrationsseite oder das
Speichern von geplanten/erledigten Prüfungsleistungen oder ausgewählten Veranstaltungen
aktivieren Sie bitte Cookies.

Die Seite sollte in allen halbwegs modernen Browsern funktionieren, inklusive den Konsolenbrowsern \texttt{w3m}
und \texttt{lynx}.

\section{Fehler oder Wünsche melden}

Über den Menüpunkt \frq Kontakt\flq\ ganz unten auf der Seite ist es möglich, mit der Studienberatung oder
mir in Kontakt zu treten. Dazu müssen Sie Ihre Email-Adresse, die Art des Anliegens und das Anliegen selbst
formulieren (die Art des Anliegens bestimmt, wohin die Email geht: bei Technischen Fragen, Wünschen,
Verbesserungsvorschlägen oder Fehlerberichten wählen Sie bitte \frq technischer Natur\flq, sonst
\frq inhaltlicher\flq). Dazu muss gegenenfalls noch eine Sicherheitsabfrage beantwortet werden, die aus
sehr einfachen mathematischen oder geographischen Fragen besteht.

\textit{Jede Email wird gelesen! Es wird um viel Feedback gebeten, da ich die Software so gut wie möglich an
die realen Bedürfnisse von Studenten ausrichten möchte!}

Ich möchte aber darum bitten zu bedenken, dass ich die Software nur als Hobby neben dem Studium betreibe
und daher manchmal etwas länger brauchen kann, um Fehler zu finden oder Wünsche einzubauen.



\section{Urheberrechte \& Danksagungen}

Die Urheberrechte sowohl der Software als auch der Dokumentation liegen bei Norman Koch.

Ich möchte mich außerdem bedanken für die freundliche Hilfe des philosophischen Institutes der Technischen
Universität Dresden, Herrn Prof. Dr. Schönrich,
Herrn Prof. Dr. Rentsch und Herrn Prof. Dr. Hiltscher. Vorallem was das Technische angeht, möchte ich mich für die 
Hilfe von  Herrn Dr. Holm Bräuer und Herrn Norbert Engemaier bedanken.
Auch herzlichen Dank an Frau Gilda Märcz, die mir sowohl im Studium als auch in der Verwaltung
immer sehr freundlich und kompetent geholfen hat.

Ohne die Hilfe, die Bugreports und die freundliche
Art aller Mitarbeiter des Institutes wäre diese Software nie entstanden.

\theendnotes

\section{Quellen}

\DeclareFieldFormat{labelnumberwidth}{---}

\printbibliography[heading=none]

\end{document}

